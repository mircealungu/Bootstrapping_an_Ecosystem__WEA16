%!TEX root=../main.tex

\section {Reflections}

Based on early user studies and the feedback we received 
regarding the individual applications we can say that such
an ecosystem shows promise. But work still has to 
be done to unequivocally show the benefits it brings 
to the learners. 

Until now, all the applications that were added, have been in one way or another supported by the creator of the ecosystem. Although all the actors could benefit from joining such an ecosystem,  the challenge of incentivizing external application developers actually join remains a future challenge. We will work on defining policies that will benefit all the players \cite{Jans09agenda}. 


Also, this is not the first ecosystem that aims to track 
the interactions of a user with data and build a better 
user model based on this; many commercial companies do it too. 
However, the goals of these companies are usually driven by economic reasons, 
while the goal of our ecosystem is mainly supporting future research and education. Indeed, one type of stakeholder that we did not have space to talk about in this paper are researchers who could have access to real world, longitudinal data.

It would seem thus, that since economical reasons do not drive the evolution of the ecosystem, we should be confronting with different problems than industrial enterprises. However, the economical 
sustainability of such a platform when hosted in academia
is not necessarily straightforward. The hosting costs of the 
core components are at the moment not significant 
but they could become large in the eventualily of 
a massive growth. 



% One other aspect that has not been discussed here
% but which will have to be dealt with at some point 
% is the privacy issue. How can we make sure that 
% private information about a reader does not leak
% between applications. 

% Finally, there is no claim or expectation that the 
% lessons derived here hold beyond the given case study. It 
% might be and it is our hope that they can be of 
% use for other learning ecosystem development
% projects, but until more such results 
% are published, this remains to be seen.



% - how to insure privacy? should everything in the profile be visible for all the apps?

