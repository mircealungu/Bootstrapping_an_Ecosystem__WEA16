%!TEX root=../main.tex

\section {Reflections}

It is still not clear when and if this ecosystem will 
take on a life on its own. Until now, all the 
applications that were added, have been in one way
or another supported by the creator. The challenge
of incentivizing external players and
to defining policies that will benefit all the 
players still remains \cite{Jans09agenda}.


Also, this is not the first ecosystem that aims to track 
the interactions of a user with data and build a better 
user model based on this; many commercial companies do it too. 
However, the goals of these companies are usually economic, 
while the goal of our ecosystem would be research and education.

It would seem thus, that since profit is not the main
goal of such an ecosystem, we should be confronting
with different problems than those enterprises where 
profit is critical. However, the economical 
sustainability of the platform when hosted in academia
is still an open question. The hosting costs of the 
core components are at the moment not significant 
but they could become large in the eventualily of 
a massive growth.


One way to do this would be to unequivocally show 
the benefits it brings to the learners. Based on 
early user studies and the feedback we received 
from individual applications we can say that such
an ecosystem shows promise. But more work has to 
be done for this. 

% One other aspect that has not been discussed here
% but which will have to be dealt with at some point 
% is the privacy issue. How can we make sure that 
% private information about a reader does not leak
% between applications. 

Finally, there is no claim or expectation that the 
lessons derived here hold beyond the given case study. It 
might be and it is our hope that they can be of 
use for other learning ecosystem development
projects, but until more such results 
are published, this remains to be seen.



% - how to insure privacy? should everything in the profile be visible for all the apps?

