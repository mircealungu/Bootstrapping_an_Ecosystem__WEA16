%!TEX root=../main.tex


\section {A Learning Ecosystem}

To address the disadvantages of the locked down nature of the existing infrastructures, we propose as a solution an open ecosystem that will combine their advantages while avoiding their limitations. The ecosystem should be built on the following principles:

\begin{itemize}

	\item The learners should be able to read anything that they find interesting. Ubiquitous translations should be offered while the context in which the translation request was made will be saved to the learners' profile. The context can be useful for further personalized exercises. To cover as many reading contexts as possible, it should be easy for new reading applications to join the ecosystem.

	\item An evolving model of the current state of the knowledge of the learner should be built by tracking and observing the users' interactions with foreign texts. The applications in the ecosystem that want to, should have access to this model.

	\item Intelligent agents should recommend reading and exercises that maximize the likelihood of encountering the most important learned words at the optimal times for every learner.

	\item The system should be open for any type of application that wants to contribute to it, either academic or industrial. The application would be able to make use of the information in the user profile as long as they contribute back to the ecosystem. 

\end{itemize}

