%!TEX root=../main.tex


\section {A Monitoring Ecosystem}

To address the disadvantages of the locked down nature of the existing infrastructures, we propose as a solution an open ecosystem that will combine their advantages while avoiding their limitations. The ecosystem should be built on the following principles:

\begin{itemize}

	\item The learners should be able to {\bf read anything that they find interesting}. Ubiquitous translations should be offered in context. 

	\item A user should have everywhere access to the words that he translated in the past and their context. Learning new words is much better done in context.

	% The context can be useful for further personalized exercises. To cover as many reading contexts as possible, it should be easy for new reading applications to join the ecosystem.

	\item An evolving model of the current state of the knowledge of the learner should be built by monitoring and observing as much as possible the users' interactions with foreign texts. The model should be available online for the applications members of the ecosystem.
	% The applications in the ecosystem that want to, should have access to this model.

	\item Intelligent agents should recommend reading and exercises that maximize the likelihood of encountering the most important learned words at the optimal times for every learner.

	\item The system should be open for any type of application that wants to contribute to it, either academic or industrial. The application would be able to make use of the information in the user profile as long as they contribute back to the ecosystem. 

\end{itemize}

We call this a {\em monitoring ecosystem} since the fact that the users activity is closely monitored is critical for the ecosystem. There can be different goals for monitoring ecosystems; in our case the goal is accelerating the learning process of factual information. 

The reasons for the various participants in this ecosystem are different, but everybody has something to gain: 

\begin{itemize}
	\item The Reader Applications. They receive information about the preferences of the user as well as his current knowledge of the various topics. They do not have to worry about storing the information about 

	\item The Trainer Applications. They receive information about the current knowledge of the user, and can better tailor exercises for them.
	
	\item The user. By using applications in the monitoring ecosystem, he has the confidence that his knowledge is mo

	\item The Core of the System. Can benefit from running anonymized analyses and discover trends and run scientific experiments.
\end{itemize}

