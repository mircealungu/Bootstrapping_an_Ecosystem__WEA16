%!TEX root=../main.tex


\section {An Open Monitoring Ecosystem}

To address the disadvantages of the locked down nature of the existing infrastructures, we propose as a solution an ecosystem consisting of a federation of applications that support a learner in accelerating vocabulary acquisition. Such an ecosystem should be built on the following principles:

\begin{description}

	\item [Free Reading.] The learners are able to read the materials they find interesting with the applications they prefer to use. When a learner encounters a word he does not understand, a high-quality, contextual translation will be offered.

	\item [Ubiquitous Monitoring.] All the encounters of a learner with foreign materials are monitored in order to build an evolving model of the current state of the knowledge of the learner. The applications that are part of the ecosystem report a users interactions with the texts to a central repository.

	\item [Accelerated Learning.] Based on the learner model intelligent agents recommend further reading materials that are both likely to be interesting to the reader and at the same time likely to maximize the retention of the most important studied words. 
	The original context of the words can be used to accelerate retention.

	\item [Openness.] The ecosystem is open to any reader or trainer application, be it open source or not, as long as it contributes back to the ecosystem. The data about a given leaner belongs to the learner themselves: they decide which applications have access their data.

\end{description}

This paper presents an ecosystem which has as a goal supporting a learner in vocabulary acquisition. However, an {\bf ubiquitous monitoring ecosystem} can have other goals as long as it represents a {\em 
	federation of applications 
		that 
			monitor a certain aspect of a user's interaction with information
			to build an evolving model of the user knowledge}.

It might appear intuitive for the reader that when the goal is accelerating the velocity of knowledge acquisition, such an ecosystem will present a network effect: the more contexts in which the reader is supported by applications from the ecosystem, the better the learner knowledge model that can be built, and the better the provided user experience.


% Move to discussion! 
% The reasons for the various participants in this ecosystem are different, but everybody has something to gain: 

% \begin{itemize}
% 	\item The Reader Applications. They receive information about the preferences of the user as well as his current knowledge of the various topics. They do not have to worry about storing the information about 

% 	\item The Trainer Applications. They receive information about the current knowledge of the user, and can better tailor exercises for them.
	
% 	\item The user. By using applications in the monitoring ecosystem, he has the confidence that his knowledge is mo

% 	\item The Core of the System. Can benefit from running anonymized analyses and discover trends and run scientific experiments.
% \end{itemize}

