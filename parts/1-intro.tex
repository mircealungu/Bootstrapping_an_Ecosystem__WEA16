%!TEX root=../main.tex

\section{Learning A New Language}

% it would be nice to talk about second language acquisition in the paper too if the term appears in the title

At any given moment millions of people are learning the vocabulary of a new language. The first steps in the acquisition of the new language are usually full of enthusiasm, but often the learner gives up once he realizes the magnitude of the task.

Indeed, once a learner has acquired the basic vocabulary of a foreign language, they are still many thousands of words away from actually mastering the new language. To improve their vocabulary they must constantly expose themselves to contexts in which the learned language is used at a level that is not too difficult but not too easy either -- that is, {\em they must study in the zone of proximal development}. Reading language textbooks is the traditional approach but more often than not textbooks being written for everybody are uninteresting for anyone.

Amazon has made a first step towards allowing the learner to read engaging texts by integrating translations and basic vocabulary exercises with their proprietary eBook reader device. Besides the limitations of being locked to a given dvice, this solution suffers also from the fact that the words being learned are locked in by Amazon. We will see later that this limits the possibilities of accelerating the speed with which vocabulary acquisition happens.

Another solution that the readers have is using Google Translate on the web. However, just as with Amazon, the translations that the user makes online are as of middle of 2016 not available outside the Google ecosystem. Thus the possibility of other applications benefiting from the knowledge of what the user is learning at the moment is also absent.

On the side of vocabulary rehearsal applications a plethora of solutions most of them being its own data silo. 

